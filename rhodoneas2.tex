\documentclass[12pt, a4paper]{article}
\usepackage[utf8]{inputenc}
\usepackage{breqn}
\usepackage{amsmath}
\usepackage{amssymb}
\usepackage{spalign}

\author{Miguel Sánchez}
\title{Angle to close a rhodonea curve}
\date{\today}

\begin{document}
\maketitle
\noindent
 We can define the rhodonea in polar coordinates as:
 \begin{equation}
  r(\theta) = \cos(k\theta)
 \end{equation}
 If we make it on cartesian coordinates we can define it as following:
 \begin{equation}
  x = \cos(k\theta)\cos(\theta)
 \end{equation}
 \begin{equation}
  y = \cos(k\theta)\sin(\theta)
 \end{equation}
 If we want the rhodonea to be closed we would need to have the last point in the exact same position of the first, so we can establish that:
\[
  \spalignsys{
    cos (k\theta)cos(\theta) = 1 ;
    cos (k\theta)sin(\theta) = 0 ;
  }
\]
Before solving the system we're going to identify the four different cases we can encounter:
\begin{itemize}
 \item $k = a$ where $a\in2\mathbb{W}$
 \item $k = a$ where $a\in\mathbb{W}\setminus2\mathbb{W}$
 \item $k = \frac{a}{b}$ where $a \in2\mathbb{W}, b \in\mathbb{W}\setminus2\mathbb{W}$
 \item $k = \frac{a}{b}$ where $a\in\mathbb{W}\setminus2\mathbb{W}, b\in2\mathbb{W}$
\end{itemize}
\newpage
\noindent
\textbf{FIRST CASE:}

\end{document}
